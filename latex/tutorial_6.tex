\documentclass[aspectratio=169, 10pt]{beamer}
\usetheme{Madrid}
\usefonttheme{professionalfonts}

\usepackage[english]{babel}
\usepackage[linguistics]{forest}
\usepackage[utf8]{inputenc}
\usepackage{algorithmic}
\usepackage{amsfonts}
\usepackage{amsmath}
\usepackage{amssymb}
\usepackage{array}
\usepackage{bookmark}
% \usepackage{boondox-cal}
\usepackage{caption}
\usepackage{colortbl}
\usepackage{csquotes}
\usepackage{graphicx}
\usepackage{hyperref}
\usepackage{lipsum}
\usepackage{lmodern}
\usepackage{mathptmx}
\usepackage{mathtools}
\usepackage{multirow}
\usepackage{pgfplots}
\usepackage{svg}
\usepackage{xcolor}

% \pgfplotsset{compat=1.17}
\usetikzlibrary{calc}

\hypersetup{
    colorlinks=true,
    linkcolor=blue,
    filecolor=blue,      
    urlcolor=blue,
}

\title{Tutorial 6}
\subtitle{Tutorial on Preprocessing}
\author{Luke Chang}
\institute{The University of Auckland}
\date{April 2021}


\begin{document}

\frame{\titlepage}

% %-------------------------------------------------------------------------------
\begin{frame}
    \frametitle{Topics}

    \tableofcontents
        
\end{frame}

%-------------------------------------------------------------------------------
\section{Data Cleaning - Noisy Data}
\begin{frame}[t]
    \frametitle{Data Cleaning - Noisy Data}

    \begin{block}{Question 1}
        Give the following data (in ascending order) for the attribute \textit{age}: 
        \[13,15,16,16,19,20,20,21,22,22,25,25,25,25,30,33,33,35,35,35,35,36,40,45,46,52,70\]
    \end{block}

    \begin{enumerate}
        \item Use \textit{smoothing by bin means} to smooth these data with 3 bins. 
        Illustrate your steps. Comment on the effect of this technique for the given data.
        \item How do you determine \textit{outliers} in the data?
        \item What other methods are there for data smoothing?
    \end{enumerate}

\end{frame}

%-------------------------------------------------------------------------------
\begin{frame}[t]
    \frametitle{Binning}
    \small

    \begin{block}{Question 1.1}
        \[13,15,16,16,19,20,20,21,22,22,25,25,25,25,30,33,33,35,35,35,35,36,40,45,46,52,70\]
        Use \textit{smoothing by bin means} to smooth these data with 3 bins.
    \end{block}
    \textbf{Step 1:} Sort the data in ascending order \\
    \textbf{Step 2:} Compute the number of samples per bin using the \textbf{equal-frequency} strategy: $27/3=9$\\
    \textbf{Step 3:} Partition into \textbf{equal-frequency} bins:

    \begin{itemize}
        \item \textbf{Bin 1:} $13,15,16,16,19,20,20,21,22$
        \item \textbf{Bin 2:} $22,25,25,25,25,30,33,33,35$
        \item \textbf{Bin 3:} $35,35,35,36,40,45,46,52,70$
    \end{itemize}

    \textbf{Step 4:} Smoothing by bin \textbf{means}:

    \begin{itemize}
        \item \textbf{Bin 1:} $18, 18, 18, 18, 18, 18, 18, 18, 18$
        \item \textbf{Bin 2:} $28, 28, 28, 28, 28, 28, 28, 28, 28$
        \item \textbf{Bin 3:} $44, 44, 44, 44, 44, 44, 44, 44, 44$
    \end{itemize}
\end{frame}

%-------------------------------------------------------------------------------
\begin{frame}[t]
    \frametitle{The Effect of Binning}

    \begin{block}{Question 1.1}
        \[13,15,16,16,19,20,20,21,22,22,25,25,25,25,30,33,33,35,35,35,35,36,40,45,46,52,70\]
        Use \textit{smoothing by bin means} to smooth these data with 3 bins.
    \end{block}

    After smoothing by bin means:
    \begin{itemize}
        \item \textbf{Bin 1:} $18, 18, 18, 18, 18, 18, 18, 18, 18$
        \item \textbf{Bin 2:} $28, 28, 28, 28, 28, 28, 28, 28, 28$
        \item \textbf{Bin 3:} $44, 44, 44, 44, 44, 44, 44, 44, 44$
    \end{itemize}

    The larger the width, the greater the effect of the smoothing.\\
    The widths for bins are 9, 13 and \textbf{35}, respectively.\\
    Binning has the greatest effect on Bin 3 ($70-44=26$).
\end{frame}

%-------------------------------------------------------------------------------
\begin{frame}[t]
    \frametitle{Remove Outliers by Clustering}

    \begin{block}{Question 1.2}
        \[13,15,16,16,19,20,20,21,22,22,25,25,25,25,30,33,33,35,35,35,35,36,40,45,46,52,70\]
        How do you determine \textit{outliers} in the data?
    \end{block}

    \textbf{Outlier analysis using Clustering:} We split the data into 3 groups, or ``clusters''.
    Once we obtain the mean and standard deviation for each cluster, 
    then we can use the 95\% rule and set the cut-off point at $2\sigma$.

    \textbf{Step 1:} Compute the absolute deviation, $\text{abs}(x - \mu)$

    \begin{itemize}
        \item \textbf{Bin 1:} $5, 3, 2, 2, 1, 2, 2, 3, 4$
        \item \textbf{Bin 2:} $6, 3, 3, 3, 3, 1, 5, 5, 7$
        \item \textbf{Bin 3:} $ 9,  9,  9,  8,  4,  1,  2,  8, 26$
    \end{itemize}

    \textbf{Step 2:} Compute means and standard deviations (We had the mean from binning.) 

    \[\sigma = \sqrt{\frac{1}{N}\sum_{N}^{i=1}(x_i-\mu)^2}\]

\end{frame}

%-------------------------------------------------------------------------------
\begin{frame}[t]
    \frametitle{Remove Outliers by Clustering}

    \begin{block}{Question 1.2}
        \[13,15,16,16,19,20,20,21,22,22,25,25,25,25,30,33,33,35,35,35,35,36,40,45,46,52,70\]
        How do you determine \textit{outliers} in the data?
    \end{block}

    \begin{table}[]
        \begin{tabular}{l|ccc}
              & Mean & $\sigma$  & $2\sigma$ \\
              \hline
              \textbf{Bin 1} & 18   & 2.9  & 6          \\
              \textbf{Bin 2} & 28   & 4.4  & 9          \\
              \textbf{Bin 3} & 44   & 10.9 & 22        
        \end{tabular}
    \end{table}

    \textbf{Step 3:} Find all data points that are above the threshold
    \begin{itemize}
        \item \textbf{Bin 1:} $5, 3, 2, 2, 1, 2, 2, 3, 4$
        \item \textbf{Bin 2:} $6, 3, 3, 3, 3, 1, 5, 5, 7$
        \item \textbf{Bin 3:} $ 9,  9,  9,  8,  4,  1,  2,  8, \textbf{26}$
    \end{itemize}

    We identify the data point ``70'' is an outlier.
\end{frame}

%-------------------------------------------------------------------------------
\begin{frame}[t]
    \frametitle{Remove Outliers by Clustering}

    \begin{block}{Question 1.2}
        \[13,15,16,16,19,20,20,21,22,22,25,25,25,25,30,33,33,35,35,35,35,36,40,45,46,52,70\]
        How do you determine \textit{outliers} in the data?
    \end{block}

    \begin{block}{Bonus Question}
    For ordinal data, if we identify a point in the cluster in the middle is an outlier.\\
    Is it actually an outlier?
    \end{block}

    \begin{example}
        We have a house price dataset in New Zealand. We cluster the data based on regions.
        In 2020, the median house price for Invercargill is $357,000$ and for Auckland is $1,030,000$.
        We find a house sold for $400,000$ in the Auckland, and it is more than $2\sigma$ below the mean.
        Then it is an outlier in the Auckland cluster.
    \end{example}

\end{frame}

%-------------------------------------------------------------------------------
\begin{frame}[t]
    \frametitle{Other Methods for Data Smoothing}

    \begin{block}{Question 1.3}
        \[13,15,16,16,19,20,20,21,22,22,25,25,25,25,30,33,33,35,35,35,35,36,40,45,46,52,70\]
        What other methods are there for data smoothing?
    \end{block}

    What tools are available?
    \begin{itemize}
        \item Binning: Done
        \item Regression: Not the best choice for 1D data.
        \item Clustering: Unsupervised learning, e.g., K-means algorithm (The example is in the notebook.)
        \item Moving average
    \end{itemize}

\end{frame}

%-------------------------------------------------------------------------------
\begin{frame}[t]
    \frametitle{Moving Average}
    \small

    \begin{block}{Question 1.3}
        \[13,15,16,16,19,20,20,21,22,22,25,25,25,25,30,33,33,35,35,35,35,36,40,45,46,52,70\]
        Smooth the data using moving average method with a fixed window size of 3.
    \end{block}

    \textbf{Step 1:} Sort the data in ascending order \\
    \textbf{Step 2:} Add padding based on the window size
    \[\underline{\textcolor{red}{13},13,15},16,16,19,20,20,21,22,22,25,25,25,25,30,33,33,35,35,35,35,36,40,45,46,52,70,\textcolor{red}{70}\]
    \textbf{Step 3:} Update each data point based on its neighbour values
    \[x'_i = \frac{x_{i-1} + x_i + x_{i+1}}{3}\]

    \[x'_1 = \frac{13+13+15}{3}=\frac{41}{3}\approx 13.7 = 14 \]

    \textbf{After smoothing} (Round to 0 d.p.)
    \[14, 15, 16, 17, 18, 20, 20, 21, 22, 23, 24, 25, 25, 27, 29, 32, 34, 34, 35, 35, 35, 37, 40, 44, 48, 56, 64\]
\end{frame}

%-------------------------------------------------------------------------------
\section{Data Transformation by Normalization}
\begin{frame}[t]
    \frametitle{Data Transformation by Normalization}

    \begin{block}{Question 2}
        Use these methods to normalize the following data points:
        \[200, 300, 400, 600, 1000\]
    \end{block}

    \begin{enumerate}
        \item Min-max normalization by setting min = 0 and max = 1
        \item Z-score normalization
        \item Z-score normalization using the mean absolute deviation instead of standard deviation
        \item Decimal scaling
    \end{enumerate}

    What are the value ranges of the following normalization methods?

\end{frame}

%-------------------------------------------------------------------------------

\begin{frame}[t]
    \frametitle{Min-max normalization}
    \small

    \begin{block}{Question 2.1}
        \[200, 300, 400, 600, 1000\]
        Min-max normalization by setting min = 0 and max = 1
    \end{block}

    \textbf{Question:} What are the value ranges of min-max normalization?\\
    You can set \textit{min} and \textit{max} to any value, but the most commonly used ranges are $[0, 1]$ and $[-1, 1]$.\\

    \vspace{1em}

    \textbf{Step 1:} Find $\text{min}=200$ and $\text{max}=1000$\\
    \textbf{Step 2:} Apply the formula to each data point
    \[x' = \frac{x - \text{min}(x)}{ \text{max}(x) - \text{min}(x)}\]

    \[0, \frac{300-200}{800}, \frac{400-200}{800}, \frac{600-200}{800}, 1\]
    \[0, \frac{1}{8}, \frac{1}{4}, \frac{1}{2}, 1\]

\end{frame}

%-------------------------------------------------------------------------------

\begin{frame}[t]
    \frametitle{Z-score Normalization}
    \small

    \begin{block}{Question 2.2}
        Apply z-score normalization to $200, 300, 400, 600, 1000$
    \end{block}

    \textbf{Question:} What are the value ranges of z-score normalization?\\
    Centred at 0. It can be any real number. However, it's unlikely to get extreme values.\\

    \vspace{1em}

    \textbf{Step 1:} Compute mean $\mu$ and standard deviation $\sigma$
    \[\mu = \frac{200+300+400+600+1000}{5}=500\]
    \[\sigma = \sqrt{\frac{1}{5}\sum_{5}^{i=1}(x_i-\mu)^2} = \sqrt{\frac{(-300)^2 + (-200)^2 + (-100)^2 + 100^2 + 500^2}{5}}= \sqrt{80000}=282.8\]

    \textbf{Step 2:} Apply the z-score formula to each data point
    \[z_i = \frac{x_i - \mu}{ \sigma}\]

    \textbf{Results:} $-1.06, -0.71, -0.35, 0.35, 1.77$ (round at 2 d.p.)
    
\end{frame}

%-------------------------------------------------------------------------------

\begin{frame}[t]
    \frametitle{Z-score Normalization Using the Mean Absolute Deviation}
    \small

    \begin{block}{Question 2.3}
        Apply z-score normalization using the \textbf{mean absolute deviation} to $200, 300, 400, 600, 1000$
    \end{block}

    \textbf{Step 1:} We already have $\mu=500$.

    \textbf{Step 2:} Compute absolute deviation
    \begin{table}[]
        \begin{tabular}{ccc}
              X & deviation  & absolute deviation \\
              \hline
              200  & -300 & 300\\
              300  & -200 & 200\\
              400  & -100 & 100\\
              600  & 100  & 100\\
              1000 & 500  & 500
        \end{tabular}
    \end{table}
    \textbf{Step 3:} Compute Mean Absolute Deviation (M.A.D)
    \[s_X  = \frac{300+200+100+100+500}{5}=\frac{1200}{5}=240 \]
    
\end{frame}

%-------------------------------------------------------------------------------

\begin{frame}[t]
    \frametitle{Z-score Normalization Using the Mean Absolute Deviation}
    \small

    \begin{block}{Question 2.3}
        Apply z-score normalization using the \textbf{mean absolute deviation} to $200, 300, 400, 600, 1000$
    \end{block}

    \textbf{Step 4:} Apply the z-score normalization using the mean absolute deviation to each data point
    
    \[z_i' = \frac{x_i - \mu}{s_X}\]
    
    \textbf{Results:} $-1.25, -0.83, -0.42,  0.42,  2.08$ (round at 2 d.p.)\\

    \vspace{1em}

    \textbf{Motivation:} Since $|x_i - \mu|$ is not squared, the effect of outliers is reduced.
\end{frame}

%-------------------------------------------------------------------------------

\begin{frame}[t]
    \frametitle{Decimal Scaling}
    \small

    \begin{block}{Question 2.4}
        Apply decimal scaling to $200, 300, 400, 600, 1000$
    \end{block}

    \textbf{Question:} What are the value ranges of decimal scaling?\\
    Within the range of $(-1, 1)$; Exclude -1 and 1.
    \vspace{1em}

    A value, $x_i$, of $X$ is normalized to $x'_i$ by computing:
    \[
        x'_i = \frac{x_i}{10^j}
    \]
    Where j is the smallest integer such that $\text{max}(|x'_i|)<1$.
    \vspace{1em}

    \begin{enumerate}
        \item Find the range of the data points: $[200, 1000]$.
        \item The maximum absolute value is 1000.
        \item Therefore, we divide each value by 10000, (i.e., $j = 4$)
    \end{enumerate}
    
    \textbf{Results:} $0.02, 0.03, 0.04, 0.06, 0.1$
    
\end{frame}

\end{document}
