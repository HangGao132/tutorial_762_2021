\documentclass[aspectratio=169, 10pt]{beamer}
\usetheme{Madrid}
\usefonttheme{professionalfonts}

\usepackage[english]{babel}
\usepackage[linguistics]{forest}
\usepackage[utf8]{inputenc}
\usepackage{algorithmic}
\usepackage{amsfonts}
\usepackage{amsmath}
\usepackage{amssymb}
\usepackage{array}
\usepackage{bookmark}
% \usepackage{boondox-cal}
\usepackage{caption}
\usepackage{colortbl}
\usepackage{csquotes}
\usepackage{graphicx}
\usepackage{hyperref}
\usepackage{lipsum}
\usepackage{lmodern}
\usepackage{mathptmx}
\usepackage{mathtools}
\usepackage{multirow}
\usepackage{pgfplots}
\usepackage{svg}
\usepackage{xcolor}

% \pgfplotsset{compat=1.17}
\usetikzlibrary{calc}

\DeclareMathOperator*{\argmax}{argmax}

\hypersetup{
    colorlinks=true,
    linkcolor=blue,
    filecolor=blue,      
    urlcolor=blue,
}

\title{Tutorial 7}
\subtitle{Tutorial on Naive Bayes}
\author{Ben Halstead, Luke Chang}
\institute{The University of Auckland}
\date{May 2021}


\begin{document}

\frame{\titlepage}

% %-------------------------------------------------------------------------------
\begin{frame}
    \frametitle{Topics}

    \tableofcontents
        
\end{frame}

%-------------------------------------------------------------------------------
\section{Multinomial Naive Bayes Classifier}
\begin{frame}[t]
\frametitle{Example 1 - Multinomial Naive Bayes Classifier}
    \begin{example}
        You come to Fiji for a holiday. 10 days later, you realise the weather forecast here isn't very accurate.
        Based on the information you gettered so far and today's weather report, you want to know ``Will it rain this afternoon?''
    \end{example}

    \begin{table}[]
        \small
        \begin{tabular}{l|llll|l}
        \textbf{Day} & \textbf{Outlook} ($O$) & \textbf{Temperature} ($T$) & \textbf{Humidity} ($H$) & \textbf{Wind} ($W$) & \textbf{Rain} ($R$) \\ \hline
        1            & Sunny            & Hot                  & High              & Weak          & True           \\
        2            & Sunny            & Hot                  & High              & Strong        & False            \\
        3            & Overcast         & Hot                  & High              & Weak          & True           \\
        4            & Rain             & Mild                 & High              & Weak          & True           \\
        5            & Rain             & Cool                 & Normal            & Weak          & True           \\
        6            & Rain             & Cool                 & Normal            & Strong        & False            \\
        7            & Overcast         & Cool                 & Normal            & Strong        & False            \\
        8            & Overcast         & Mild                 & High              & Strong        & True           \\
        9            & Sunny            & Cool                 & Normal            & Weak          & False            \\
        10           & Rain             & Mild                 & Normal            & Weak          & False            \\ \hline
        11           & Sunny            & Mild                 & Normal            & Strong        & ?            
        \end{tabular}
    \end{table}

\end{frame}

%-------------------------------------------------------------------------------
\begin{frame}[t]
\frametitle{Example 1 - Formulate the Problem}
    \begin{itemize}
        \item Attribute: \textit{Outlook} ($O$), \textit{Temperature} ($T$), \textit{Humidity} ($H$), \textit{Wind} ($W$).
        \item Output: \textit{Rain} ($R$) - Binary classification problem
        \item How can we formulate this task?
        \pause
            \begin{itemize}
                \item The probability of a given output $r \in \{\text{True}, \text{False}\}$ is:
                    $$P(R=r| O,T,H,W)$$
                \pause
                \item We want to predict the label with the highest probability.
                    $$R = \argmax_{r \in \{\text{T}, \text{F}\}} P(R=r| O,T,H,W)$$
            \end{itemize}
    \end{itemize}
\end{frame}

%-------------------------------------------------------------------------------
\begin{frame}[t]
    \frametitle{Example 1 - Formulate the Problem}
        \begin{itemize}
            \item Bayes Theorem: 
            \[ P(Y|X) = \frac{P(X|Y) P(Y)}{P(X)} \]
            \item Terminology -- Prior: $P(Y)$, Likelihood: $P(X|Y)$, Posterior: $P(Y|X)$, Marginal Probability: $P(X)$.
            \item How to rewrite this expression using Bayes Theorem?
            \pause
            \[ R = \argmax_{r \in \{\text{T}, \text{F}\}} P(R=r| O,T,H,W) \]
            \[ R = \argmax_{r \in \{\text{T}, \text{F}\}} \frac{P(O,T,H,W| R=r) P(R=r)}{P(O,T,H,W)} \]
            \item How can we simplify this problem?
            \pause
            \\ $P(Y|X)$ is proportional to: $P(Y|X) \propto P(X|Y) P(Y)$
            \[ R = \argmax_{r \in \{\text{T}, \text{F}\}} P(O,T,H,W| R=r) P(R=r) \]
            \item The marginal Probability $P(O,T,H,W)$ is omitted. If we want to know the probability, we can normalise all possible outcomes.
        \end{itemize}
    \end{frame}

%-------------------------------------------------------------------------------
\begin{frame}[t]
    \frametitle{Example 1 - Calculate the Prior $P(R)$}
        \begin{itemize}
            \item 10 observations
            \item 5 days was raining.
            \item 5 days wasn't.
            \pause
            \item $P(R=\text{True}) = \frac{5}{10} = 0.5$
            \item $P(R=\text{False}) = \frac{5}{10} = 0.5$
        \end{itemize}
    
    \pause
    \begin{example}
        \textbf{Gambler's fallacy:} Toss a coin, I see 20 head showed up in a row.
        The next time it must land on tail, since the probability of a coin landing 21 times is too low, $0.5^{21}$.
    \end{example}
    \pause
    Problem:
    \begin{itemize}
        \item Frequentist Statistics: Each tail is independent.
        \item $P(Y) \neq P(Y|X)$
        \item Bayes Theorem: If the same coin is used, the more evidence you collect, the more likely the coin is loaded.
    \end{itemize}
\end{frame}

%-------------------------------------------------------------------------------
\begin{frame}[t]
    \frametitle{Example 1 - Calculate the Likelihood}
    \begin{itemize}
        \item The first part of the formula, $P(O,T,H,W|R=r)$, is the \textit{likelihood}.
        \item It describes how \textit{likely} an event occurs, given the outcome.
        \item How do we calculate the \textit{likelihood}?
        \pause
        \item We could try to calculate $P(O,T,H,W|R=r)$ directly. What is the problem with this approach?
            \begin{itemize}
                \item We need to compute all possible combinations.
                \item We have: $3\times3\times2\times2 = 36$ different combinations of $O,T,H,W$ \textit{per label}.
                \item We only have 10 observations -- not enough to calculate probabilities for all combinations.
            \end{itemize}
        \item Naive Bayes makes the assumption that all attributes are independent: allowing us to calculate the likelihood as:
            \[
                P(O,T,H,W|R=r) = P(O|R=r)p(T|R=r)p(H|R=r)p(W|R=r)
            \]    
    \end{itemize}
\end{frame}

%-------------------------------------------------------------------------------
\begin{frame}[t]
    \frametitle{Example 1 - Calculate the Likelihood}
    \begin{table}[]
        \small
        \begin{tabular}{l|llll|l}
        \textbf{Day} & \textbf{Outlook} ($O$) & \textbf{Temperature} ($T$) & \textbf{Humidity} ($H$) & \textbf{Wind} ($W$) & \textbf{Rain} ($R$) \\ \hline
        1            & Sunny            & Hot                  & High              & Weak          & True           \\
        3            & Overcast         & Hot                  & High              & Weak          & True           \\
        4            & Rain             & Mild                 & High              & Weak          & True           \\
        5            & Rain             & Cool                 & Normal            & Weak          & True           \\
        8            & Overcast         & Mild                 & High              & Strong        & True           \\
        \end{tabular}
    \end{table}

    $$\sum P(X=x_i| R=r) = 1$$

    \begin{columns}
        \begin{column}{0.5\textwidth}
           \begin{itemize}
               \item $P(O=\text{Sunny} | R=\text{T}) = \frac{1}{5}$
               \item $P(O=\text{Overcast} | R=\text{T}) = \frac{2}{5}$
               \item $P(O=\text{Rain} | R=\text{T}) = \frac{2}{5}$
           \end{itemize}
           \vspace{0.5em}
           \begin{itemize}
            \item $P(T=\text{Hot} | R=\text{T}) = \frac{2}{5}$
            \item $P(T=\text{Mild} | R=\text{T}) = \frac{2}{5}$
            \item $P(T=\text{Cool} | R=\text{T}) = \frac{1}{5}$
        \end{itemize}
        \end{column}
        \begin{column}{0.5\textwidth}  %%<--- here
            \begin{itemize}
                \item $P(H=\text{High} | R=\text{T}) = \frac{4}{5}$
                \item $P(H=\text{Normal} | R=\text{T}) = \frac{1}{5}$
            \end{itemize}
            \vspace{0.5em}
            \begin{itemize}
                \item $P(T=\text{Strong} | R=\text{T}) = \frac{1}{5}$
                \item $P(T=\text{Weak} | R=\text{T}) = \frac{4}{5}$
            \end{itemize}        
        \end{column}
    \end{columns}
\end{frame}

%-------------------------------------------------------------------------------
\begin{frame}[t]
    \frametitle{Example 1 - Calculate the Likelihood}
    \begin{table}[]
        \small
        \begin{tabular}{l|llll|l}
        \textbf{Day} & \textbf{Outlook} ($O$) & \textbf{Temperature} ($T$) & \textbf{Humidity} ($H$) & \textbf{Wind} ($W$) & \textbf{Rain} ($R$) \\ \hline
        2            & Sunny            & Hot                  & High              & Strong        & False            \\
        6            & Rain             & Cool                 & Normal            & Strong        & False            \\
        7            & Overcast         & Cool                 & Normal            & Strong        & False            \\
        9            & Sunny            & Cool                 & Normal            & Weak          & False            \\
        10           & Rain             & Mild                 & Normal            & Weak          & False            \\
        \end{tabular}
    \end{table}

    \begin{columns}
        \begin{column}{0.5\textwidth}
           \begin{itemize}
               \item $P(O=\text{Sunny} | R=\text{F}) = \frac{2}{5}$
               \item $P(O=\text{Overcast} | R=\text{F}) = \frac{1}{5}$
               \item $P(O=\text{Rain} | R=\text{F}) = \frac{2}{5}$
           \end{itemize}
           \vspace{0.5em}
           \begin{itemize}
            \item $P(T=\text{Hot} | R=\text{F}) = \frac{1}{5}$
            \item $P(T=\text{Mild} | R=\text{F}) = \frac{1}{5}$
            \item $P(T=\text{Cool} | R=\text{F}) = \frac{3}{5}$
        \end{itemize}
        \end{column}
        \begin{column}{0.5\textwidth}  %%<--- here
            \begin{itemize}
                \item $P(H=\text{High} | R=\text{F}) = \frac{1}{5}$
                \item $P(H=\text{Normal} | R=\text{F}) = \frac{4}{5}$
            \end{itemize}
            \vspace{0.5em}
            \begin{itemize}
                \item $P(T=\text{Strong} | R=\text{F}) = \frac{3}{5}$
                \item $P(T=\text{Weak} | R=\text{F}) = \frac{2}{5}$
            \end{itemize}        
        \end{column}
    \end{columns}
\end{frame}

%-------------------------------------------------------------------------------
\begin{frame}[t]
    \frametitle{Example 1 - Calculate the Posterior}
    \begin{table}[]
        \small
        \begin{tabular}{l|llll|l}
        \textbf{Day} & \textbf{Outlook} ($O$) & \textbf{Temperature} ($T$) & \textbf{Humidity} ($H$) & \textbf{Wind} ($W$) & \textbf{Rain} ($R$) \\ \hline
        11           & Sunny            & Mild                 & Normal            & Strong        & ?            
        \end{tabular}
    \end{table}

    Now we have:
    \[ R = \argmax_{r \in \{\text{T}, \text{F}\}} P(O| R=r)P(T| R=r)P(H| R=r)P(W| R=r) P(R=r) \]

    \begin{align*}
        P(R=\text{T}|O,T,H,W) & \propto P(O=\text{S}| R=\text{T})P(T=\text{M}| R=\text{T})P(H=\text{N}| R=\text{T})P(W=\text{S}| R=\text{T}) P(R=\text{T}) \\
        & \propto \frac{1}{5} \cdot \frac{2}{5} \cdot \frac{1}{5} \cdot \frac{1}{5} = 0.0032\\
    \end{align*}

    \begin{align*}
        P(R=\text{F}|O,T,H,W) & \propto P(O=\text{S}| R=\text{F})P(T=\text{M}| R=\text{F})P(H=\text{N}| R=\text{F})P(W=\text{S}| R=\text{F}) P(R=\text{F}) \\
        & \propto \frac{2}{5} \cdot \frac{1}{5} \cdot \frac{4}{5} \cdot \frac{3}{5} = 0.0384\\
    \end{align*}
\end{frame}

%-------------------------------------------------------------------------------
\begin{frame}[t]
    \frametitle{Example 1 - How Likely?}
    \begin{table}[]
        \small
        \begin{tabular}{l|llll|l}
        \textbf{Day} & \textbf{Outlook} ($O$) & \textbf{Temperature} ($T$) & \textbf{Humidity} ($H$) & \textbf{Wind} ($W$) & \textbf{Rain} ($R$) \\ \hline
        11           & Sunny            & Mild                 & Normal            & Strong        & ?            
        \end{tabular}
    \end{table}

    \begin{align*}
        P(R=\text{T}|O,T,H,W) & = \frac{0.0032}{P(O,T,H,W)}\\
        P(R=\text{F}|O,T,H,W) & = \frac{0.0384}{P(O,T,H,W)}\\
        P(R=\text{T}|O,T,H,W) & = \frac{0.0032}{0.0032 + 0.0384} = 0.08\\
        P(R=\text{F}|O,T,H,W) & = \frac{0.0384}{0.0032 + 0.0384} = 0.92\\
    \end{align*}

    Given the information, there is $92\%$ to not rain.
\end{frame}

\end{document}


